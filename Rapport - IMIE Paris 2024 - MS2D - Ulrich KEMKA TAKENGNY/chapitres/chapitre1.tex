\section{Présentation du centre de recherche}
\label{chap:sectionone}
\sloppy


\subsection{Inria : métiers et chiffres clés.}

L’Inria est l’institut national de recherche en sciences et technologies du numérique, crée en 1967, il dispose de 11 centres et plus de 20 antennes et emploie 2600 personnes. La recherche de rang mondial, l’innovation technologique et le risque entrepreneurial constituent son ADN.
Au sein de 215 équipes-projets en général communes avec des partenaires académiques, plus de 3 900 chercheurs et ingénieurs y explorent des voies nouvelles. 900 personnels d’appui à la recherche et à l’innovation contribuent à faire émerger et grandir des projets scientifiques ou entrepreneuriaux qui impactent le monde.

L’institut fait appel à de nombreux talents dans plus d’une quarantaine de métiers différents, travaille avec de nombreuses entreprises et a accompagné la création de plus de 180 start-ups.
Inria soutient la diversité des voies de l’innovation : de l’édition open source de logiciels à la création de startups technologiques (Deeptech).


\subsection{Le service de l’apprenti : Mission Défense et Sécurité}

Le renforcement des partenariats avec la sphère Sécurité et Défense de l’État est une priorité stratégique de l’Inria. C’est de ce contexte qu’est né l’institut. Créé en mars 2020 et dirigé par Frédérique Segond, la Mission Défense et Sécurité a pour objectif le soutient des politiques gouvernementales qui visent la souveraineté et l’autonomie stratégique numérique de l’Etat français, voire européen. Elle fédère tous les projets sécurité défense d’Inria bientôt de toute la France.
L’équipe en pleine croissance est actuellement composée de seize personnes ayant chacun un rôle bien définit avec le soutien des intervenants externes à la mission.

\section{Principales activités}


Sous la supervision du responsable informatique, j'ai développé divers outils pédagogiques et technologiques visant à faciliter la prise en main des applications et à enrichir la compréhension des scénarios d'exercice de simulation. Mon travail a contribué à l'élaboration et à l'optimisation de nouvelles technologies au sein de l'équipe.

J'ai principalement participé au développement et à l'évolution de la plateforme IntelLab, un projet clé permettant de simuler des scénarios de sécurité variés.
Nous avons travaillé en mode projet avec la méthode Agile, ce qui nous a permis d'assurer une gestion efficace et un suivi optimal des projets.

\noindent Mes contributions se sont réparties sur trois projets distincts :

\begin{itemize}\addtolength{\itemsep}{-0.35\baselineskip}%
  \item \textbf{Développement de l'application web IntelLab }: J'ai pris en charge le développement des interfaces web Frontend et Backend de la plateforme, intégrant les dernières technologies pour garantir une performance et une expérience utilisateur optimales.
  \item \textbf{Mise en place d'un serveur de messagerie} : En parallèle, j'ai contribué au déploiement d'une plateforme dédiée à la simulation de scénarios de sécurité économique, centrée sur la gestion d'un serveur de messagerie. Cette plateforme est utilisée dans des exercices de simulation reproduisant des environnements d'entreprises tels que des startups ou des équipes de recherche.
  \item \textbf{Détection et reconnaissance, proche du temps réel, de véhicules militaires sur des images et vidéos} : J'ai participé à un projet visant à développer et affiner des algorithmes de Deep Learning pour la détection et la reconnaissance en temps quasi réel de véhicules militaires dans des images et vidéos.
\end{itemize}

\noindent Dans le cadre de ces projets, mes responsabilités incluent :

\begin{itemize}\addtolength{\itemsep}{-0.35\baselineskip}%
  \item \textbf{Développement web Frontend et Backend} : Conception et implémentation des fonctionnalités sur les deux volets de l'application.
  \item \textbf{Développement d'algorithmes génératifs} : Création et intégration d'algorithmes destinés à enrichir les jeux de données pour améliorer les performances des modèles.
  \item \textbf{Rédaction des tests} : Rédaction et exécution de tests pour garantir la fiabilité et la robustesse des solutions développées.
  \item \textbf{Mise en production} : Déploiement des solutions dans l'environnement de production, assurant leur bon fonctionnement et leur maintenance.
  \item \textbf{Maintenance des applications et de leurs infrastructures} : Assurer la stabilité et la disponibilité continue des applications en identifiant et résolvant les problèmes techniques rapidement.
\end{itemize}

