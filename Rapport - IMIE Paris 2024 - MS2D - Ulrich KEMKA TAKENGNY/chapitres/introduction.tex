


\markboth{\MakeUppercase{Introduction}}{}%
\addcontentsline{toc}{chapter}{Introduction}%

%Welcome to \Ac{ITBS}. ~\\
%Again, welcome to \Ac{ITBS}. ~\\
%Your introduction goes here. ~\\

Voici une référence à l'image de la Figure \ref{fig:test} page \pageref{fig:test} et une autre vers la partie \ref{chap:2} page \pageref{chap:2}.
On peut citer un livre\, \cite{caillois1} et on précise les détails à la fin du rapport dans la partie références.
Voici une note\,\footnote{Texte de bas de page} de bas de page\footnote{J'ai bien dit bas de page}. Nous pouvons également citer l'Algorithme , la Définition \ref{def1}, le Théorème \ref{theo1} ou l'Exemple \ref{exo1}...\\

Le document est détaillé comme suit : le chapitre \ref{chap:chapterone} introduit le cadre général de ce travail. Il s'agit de présenter l'entreprise d'accueil et de détailler la problématique. Le chapitre \ref{chap:2} introduit les données ainsi que les modèles choisies.\\


Dans le cadre de notre formation et en tant qu’alternant, nous sommes partagés entre les
enseignements à l’école et les enseignements en entreprise, ceci pour faciliter notre immersion
dans le monde du travail. Pour mon alternance, j’ai intégré le centre de recherche Inria, dans le département Mission Défense et Sécurité en tant que Développeur Web
Au cours de cette expérience professionnelle, j’ai participé au développement d'une application web permettant contenant des scénarios permettant de simuler l'exploitation du renseignement d'intérêt militaire,
au déploiement d'un mail serveur simulant un scénario de sécurité économique dans un environnement d'entreprise (startup, équipe projet recherche) et
à l'optimisation des Algorithmes de Deep Learning pour la Détection et la Reconnaissance en Temps Réel de Véhicules  Militaires dans des Images et Vidéos
Ces projets dont l’objectif d’une part à faire appréhender aux académiques et aux entreprises les problèmes
concrets rencontrés par les opérationnels afin d’y proposer des solutions communes, et
d’autre part, de permettre d’expérimenter les solutions sur la base de procédures de tests
« opérationnels » constitueront la pierre angulaire notre rapport.
