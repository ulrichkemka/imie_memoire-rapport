\markboth{\MakeUppercase{Introduction}}{}%
\addcontentsline{toc}{chapter}{Introduction}%
\sloppy

Dans le cadre de ma formation en alternance, j'ai eu l'opportunité de participer à plusieurs projets au sein du centre de recherche Inria, plus précisément dans le département Mission Défense et Sécurité. Mon rôle en tant que Développeur Web m'a permis de contribuer à des initiatives qui combinent des approches avancées en technologie web et en intelligence artificielle pour répondre à des problématiques concrètes liées à la sécurité et à la défense.

Au cours de cette expérience professionnelle, j'ai été impliqué dans le développement de l'application web du projet IntelLab. Cette plateforme permet de simuler divers scénarios de sécurité, dont l'exploitation du renseignement d'intérêt militaire et la détection des signaux faibles précurseurs d'actions d'ingérence.

En parallèle, j'ai participé à la mise en place d'une autre plateforme du projet IntelLab spécifique au scénario de type sécurité économique, qui se focalise sur le déploiement d'un serveur de messagerie. Cette plateforme est utilisé pour des exercices simulant des scénarios de sécurité économique, recréant des environnements d'entreprises telles que des startups ou des équipes de recherche.

Enfin, j'ai contribué au projet de détection et reconnaissance (DetReco), proche du temps réel, de véhicules militaires sur des images et vidéos. Ce projet vise à optimiser les algorithmes de Deep Learning pour la détection et la reconnaissance en temps réel de véhicules militaires dans des images et vidéos.

Ces trois projets, dont l'objectif principal est de faire le lien entre la recherche académique et les besoins opérationnels des acteurs de la défense, constituent la pierre angulaire de ce rapport. Ils illustrent comment des solutions technologiques avancées peuvent être développées, testées et optimisées pour répondre à des défis spécifiques dans le domaine de la sécurité et de la défense.

