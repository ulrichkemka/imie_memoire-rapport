
\section{Difficultés rencontrées}
Pendant le développement des différents projets, plusieurs défis techniques et organisationnels ont été rencontrés.

Le projet \textbf{IntelLab} a nécessité une refonte complète de l'application initiale pour s'adapter à de nouveaux scénarios, ce qui a révélé des limitations dans le code existant, ainsi que des problèmes de compatibilité lors de l'intégration de nouvelles fonctionnalités.

La mise en place d'une infrastructure de serveur de messagerie a été compliquée par la nécessité de configurer manuellement les adresses IP et les noms de domaine, ce qui était fastidieux et sujet à erreurs.

Pour le projet DetReco, la collecte de données d'entraînement suffisantes et représentatives a posé un défi, en particulier pour certaines classes de véhicules militaires sous-représentées.

\section{Solutions apportées}
Pour résoudre ces difficultés, plusieurs stratégies ont été mises en place.

La refonte de l'application IntelLab a été réalisée en adoptant une approche modulaire qui a facilité l'intégration de nouvelles fonctionnalités et amélioré la maintenabilité du code.

Concernant le serveur de messagerie, l'utilisation de \textbf{Mailu et Docker} a permis de containeriser les différents services, rendant le déploiement plus flexible et automatisé, et réduisant ainsi la complexité liée à la configuration manuelle.

Dans le cadre du projet DetReco, des techniques d'augmentation des données ont été utilisées pour compenser le manque d'images annotées, permettant ainsi d'améliorer la précision du modèle de détection des véhicules militaires.

\section{Leçons apprises}
Ces projets ont permis de tirer plusieurs enseignements précieux.

Tout d'abord, l'importance de l'agilité dans la gestion de projets complexes a été confirmée, particulièrement dans un contexte où les priorités et les besoins évoluent rapidement.

Ensuite, la nécessité de disposer d'une infrastructure flexible et scalable s'est avérée cruciale pour garantir la réussite du projet, comme l'a montré l'utilisation de Docker pour la gestion de nos infrastructures.

Enfin, l'expérience avec le projet DetReco a mis en lumière l'importance de la qualité et de la diversité des données d'entraînement pour le développement de modèles d'apprentissage automatique performants, soulignant ainsi la nécessité de bien planifier cette étape dès le début d'un projet.
