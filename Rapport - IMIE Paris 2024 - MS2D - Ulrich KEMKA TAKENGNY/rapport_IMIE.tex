%% @Author: KEMKA TAKENGNY Ulrich
%  @Date:   2024-08
%% @Class:  PFE IMIE-PARIS.

\documentclass[a4paper, oneside, 12pt, final]{extreport}
% Pour utiliser Times New Roman
\usepackage{mathptmx}
% \usepackage[french]{babel} % Configure le document pour le français
\usepackage{graphicx}

\parindent 1cm
\usepackage{makeidx}
\makeindex
%\usepackage[english]{babel}
\usepackage[lined,boxed,commentsnumbered, english, ruled,vlined,linesnumbered, french]{algorithm2e}
\usepackage{amsthm}
\newtheorem{theorem}{Theorem}[chapter]
\newtheorem{definition}{Definition}[chapter]
\newtheorem{exemple}{Example}[chapter]


%\usepackage[nottoc]{tocbibind}
%\addcontentsline{toc}{section}{References}

\providecommand{\keywords}[1]{\textbf{\textit{Mots clés---}} #1}
\providecommand{\keywordss}[1]{\textbf{\textit{Keywords---}} #1}

\usepackage{etoolbox}
%\makeatletter
%\patchcmd{\thebibliography}{%
%  \chapter*{\bibname}\@mkboth{\MakeUppercase\bibname}%{\MakeUppercase\bibname}}{%
%  \section{References}}{}{}
%\makeatother

% Pour l'interligne
\usepackage{setspace}

% Configurer l'interligne à 1.5
\onehalfspacing

\usepackage[nottoc]{tocbibind}

\textwidth 16.5cm
\textheight 24cm
\topmargin -1cm
\oddsidemargin -0.3cm

% set font encoding for PDFLaTeX or XeLaTeX
\usepackage{ifxetex}
\ifxetex
  \usepackage{fontspec}
\else
  \usepackage[T1]{fontenc}
  \usepackage[utf8]{inputenc}
  \usepackage{lmodern}
\fi


% Enable SageTeX to run SageMath code right inside this LaTeX file.
% documentation: http://mirrors.ctan.org/macros/latex/contrib/sagetex/sagetexpackage.pdf
%\usepackage{sagetex}


\newcommand{\reportTitle} {
  \textsc{Projet de Fin d'\'etudes}
}

\newcommand{\reportAuthor} {
  \textbf{Ulrich KEMKA TAKENGNY}
}

\newcommand{\reportSubject} {
  Développement et optimisation des plateformes de simulation pour les scénarios de sécurité et défense : \textit{IntelLab} 
}


\newcommand{\studyDepartment} {%
  Entreprise d'accueil :
  
}

\newcommand{\IMIE} {%
\\ Ecole Supérieure d'informatique IMIE-Paris
}


\newcommand{\AU} {
\centering \textbf{Année Académique 2023-2024}
}


\newcommand{\specialcell}[1]{%
  \begin{tabularx}{\textwidth}{@{}X@{}}#1\end{tabularx}%
}

%%%%%%%%%%%%%%%%%%%%%%%%%%%%%%%%%%%%%%%%%%%%%%%%%%%%%%%
% Add your own commands here
%%%%%%%%%%%%%%%%%%%%%%%%%%%%%%%%%%%%%%%%%%%%%%%%%%%%%%%
\newcommand{\MyCommand} {%
  Does nothing really%
}

\renewcommand{\contentsname}{Table des matières}
\renewcommand{\listfigurename}{Liste des Figures}
\renewcommand{\listtablename}{Liste des Tableaux}

% used in maketitle
\title{\reportSubject}
\author{\reportAuthor}

% Enable SageTeX to run SageMath code right inside this LaTeX file.
% documentation: http://mirrors.ctan.org/macros/latex/contrib/sagetex/sagetexpackage.pdf
%\usepackage{sagetex}

%\hypersetup{
%  pdftitle={\reportTitle~-~\reportSubject},%
%  pdfauthor={\reportAuthor},%
%  pdfsubject={\reportSubject},%
%  pdfkeywords={report} {internship} {pfe} {enis}
%}

\usepackage{graphics}
\usepackage{graphicx}


\usepackage[acronym,toc,section=chapter]{glossaries}
\makeglossaries

\newacronym{abc}{ABC}{A contrived acronym}
\newacronym{efg}{EFG}{Another acronym}
\newacronym{svm}{SVM}{Support Vector Machines}

\pagenumbering{roman} 

\usepackage[utf8]{inputenc}
%\usepackage[french]{babel}

\begin{document}
\thispagestyle{empty}
\begin{titlepage}
  \begin{center}

    %%%%%%%%%%%%%%%%%%%%%%%%%%%%%%%%%%%%%%%%%%%%%%%
    % THE HEADER
    %%%%%%%%%%%%%%%%%%%%%%%%%%%%%%%%%%%%%%%%%%%%%%%

    \includegraphics[scale=0.7]{./images/inr_logo_rouge.png}
    \hfill
    \includegraphics[scale=0.4]{./images/logonoir.png}

    \vspace{3cm}



    %%%%%%%%%%%%%%%%%%%%%%%%%%%%%%%%%%%%%%%%%%%%%%%
    % THE PAGE CONTENT
    %%%%%%%%%%%%%%%%%%%%%%%%%%%%%%%%%%%%%%%%%%%%%%%

    % \vspace{10pt} {%
    %   \renewcommand*{\familydefault}{\defaultFont}
    %   \fontsize{46pt}{46pt}\selectfont
    %   % MEMOIRE\\%
    %   %\reportTitle{}%\\\textsc{Report}\\%
    % }


    \textbf{\textit{Rapport d'activité de fin d'alternance}}\\

    \vspace{5pt}
    {\textit{Réalisé par}}\\
    \vspace{10pt} {%
      \fontsize{14pt}{14pt}\selectfont
      {\bfseries\Large\sc \reportAuthor}\\
    }

    \vspace{10pt}

    \vspace{5pt} {
      \renewcommand*{\familydefault}{\defaultFont}
      \fontsize{27pt}{27pt}\selectfont%
      \rule{0.5\textwidth}{.4pt}\\
      \vspace{10pt}
      \reportSubject{}\\%
      \vspace{10pt}
      \rule{0.5\textwidth}{.4pt}
    }


    \vspace{30pt}
    {Diplôme préparé : \textbf{\large Manager de Solutions Digitales et Data}}\\
    %Soutenu le \dateSoutenance, devant la commission d'examen:\\
    \vspace{46pt}


    Sous la supervision de :  \textbf{Jonas RENAULT}, \textit{Chef de projet informatique}\\
    \vspace{50pt}

  \end{center}
  \vspace{30pt}
  \AU\\
\end{titlepage}

% ###############################
% # HELP COMMANDS               #
% ###############################
%
% -1 \part{part}
%  0 \chapter{chapter}
%  1 \section{section}
%  2 \subsection{subsection}
%  3 \subsubsection{subsubsection}
%  4 \paragraph{paragraph}
%  5 \subparagraph{subparagraph}

\markboth{\MakeUppercase{Abstract}}{}
\addcontentsline{toc}{chapter}{Abstract}
\sloppy

The present work is part of a graduation project carried out within the company  in order to obtain the national diploma of engineer at the \IMIE. This project's objective is to design and implement a ... \\


%%%%%%%%%%%%%%%%%%%%%%%%%%%%%%%%%%%%%%%%%%%%%%%%%%%%%%%
% Divers chapitres
%%%%%%%%%%%%%%%%%%%%%%%%%%%%%%%%%%%%%%%%%%%%%%%%%%%%%%%

\tableofcontents
% \addcontentsline{toc}{chapter}{Table des matières}

\listoffigures
% \addcontentsline{toc}{chapter}{Liste des Figures}

\listoftables
% \addcontentsline{toc}{chapter}{Liste des Tableaux}

\listofalgorithms
\addcontentsline{toc}{chapter}{Liste des algorithmes}

\cleardoublepage

\newpage
\pagenumbering{arabic}
\chapter*{Introduction}
\label{chap:general_intorduction}
\chapter{Introduction}
\label{chap:intorduction}
\markboth{\MakeUppercase{Introduction}}{}
\sloppy

Aujourd’hui, les données sont souvent comparées au pétrole du XXIe siècle.
Chaque jour, plusieurs téraoctets de données sont stockés, constituant le carburant essentiel de la technologie moderne et de l'intelligence artificielle (IA) en particulier.
Cette abondance de données rend possible le développement d'algorithmes sophistiqués, notamment dans le domaine du deep learning, qui permettent de traiter, d'analyser et d'interpréter des informations complexes avec une précision sans précédent.

Malgré l'abondance de données disponibles, de nombreuses organisations, notamment dans le secteur de la défense, peinent à accéder à certaines catégories d'informations essentielles au développement de leurs systèmes d'intelligence artificielle.
Cette difficulté est principalement liée à la rareté et à la nature confidentielle des données spécifiques, telles que celles relatives à la reconnaissance optique, particulièrement dans des contextes de sécurité nationale.
Cela représente un défi majeur pour les systèmes automatisés de détection et de reconnaissance, qui doivent fonctionner efficacement même avec des données restreintes.


Dans ce contexte, une question clé se pose : dans quelle mesure le deep learning peut-il améliorer la détection et la reconnaissance en temps réel de véhicules militaires dans des images et vidéos, en surmontant les obstacles liés à la rareté des données et à la diversité des environnements d'observation ?
L’objectif de ce mémoire est d'explorer cette question en profondeur, en étudiant l’état de l’art des méthodes et algorithmes existants et en proposant des améliorations pour relever ces défis.

Ce travail s’inscrit dans le cadre du projet DetReco (Détection et reconnaissance de véhicules militaires sur des images et vidéos), un projet visant à optimiser les algorithmes de deep learning pour la détection et la reconnaissance en temps réel de véhicules militaires.
Le projet repose sur la création et l’amélioration de jeux de données spécifiques, ainsi que sur le fine-tuning de modèles de détection d’objets.
L'intégration de modèles génératifs, tels que ceux basés sur \textit{stable diffusion} , pour augmenter la diversité des données d'entraînement, constitue également une part importante de ce projet.

Les résultats de ce travail pourraient non seulement améliorer les capacités des systèmes de défense en matière de surveillance et de reconnaissance, mais aussi ouvrir la voie à de nouvelles recherches sur l'application des techniques de deep learning dans des domaines où les données sont rares ou difficiles d'accès.
Par conséquent, ce mémoire vise à contribuer à l'amélioration des algorithmes de détection et de reconnaissance dans des contextes critiques, tout en jetant les bases pour des développements futurs dans le domaine de l'intelligence artificielle appliquée à la défense.


\newpage

\section{Présentation de l'entreprise}
\subsection{Centre de recherche Inria}
L’Inria est l’institut national de recherche en sciences et technologies du numérique, crée en 1967, il dispose de 11 centres et plus de 20 antennes et emploie 2600 personnes. La recherche de rang mondial, l’innovation technologique et le risque entrepreneurial constituent son ADN.
Au sein de 215 équipes-projets en général communes avec des partenaires académiques, plus de 3 900 chercheurs et ingénieurs y explorent des voies nouvelles. 900 personnels d’appui à la recherche et à l’innovation contribuent à faire émerger et grandir des projets scientifiques ou entrepreneuriaux qui impactent le monde.

L’institut fait appel à de nombreux talents dans plus d’une quarantaine de métiers différents, travaille avec de nombreuses entreprises et a accompagné la création de plus de 180 start-ups.
Inria soutient la diversité des voies de l’innovation : de l’édition open source de logiciels à la création de startups technologiques (Deeptech).


\subsection{Département Défense et Sécurité}
Le renforcement des partenariats avec la sphère Sécurité et Défense de l’État est une priorité stratégique de l’Inria. C’est de ce contexte qu’est né le département. Créé en mars 2020 et dirigé par Frédérique Segond, la Mission Défense et Sécurité a pour objectif le soutient des politiques gouvernementales qui visent la souveraineté et l’autonomie stratégique numérique de l’Etat français, voire européen. Elle fédère tous les projets sécurité défense d’Inria bientôt de toute la France.
L’équipe en pleine croissance est actuellement composée de seize personnes ayant chacun un rôle bien définit avec le soutien des intervenants externes à la mission.

\newpage
\section{Contexte et problématique}

Le contexte de ce mémoire s’inscrit dans le cadre d'une collaboration entre l'équipe STARS du centre de Sophia-Antipolis, la Direction Générale de l’Armement Techniques Terrestres (DGA TT), et le département Défense et Sécurité de l’Inria.
Cette collaboration a pour objectif de répondre à un besoin stratégique : la détection et la reconnaissance, en temps réel, de véhicules militaires dans des images et des vidéos.

Les systèmes d'intelligence artificielle dédiés à la détection et à la reconnaissance d'objets ont connu une expansion rapide ces dernières années.
Dans le secteur de la défense, il est crucial que ces technologies allient précision et rapidité pour assurer la sécurité nationale.
Cependant, les environnements militaires présentent des défis spécifiques pour ces algorithmes : les véhicules militaires peuvent être camouflés, partiellement cachés, ou se trouver dans des conditions de faible visibilité.
Par ailleurs, les données utilisées pour entraîner ces systèmes sont souvent limitées en termes de quantité et de diversité, en raison de leur nature confidentielle et des restrictions d'accès aux informations militaires.

Dans ce contexte, l'application des modèles de deep learning à la reconnaissance des véhicules militaires représente une avancée significative.
Cependant, les méthodes traditionnelles de deep learning rencontrent des limitations lorsqu'elles sont appliquées à ce domaine spécifique.
Par exemple, les bruits et les interférences sur les images, la rareté des données d'entraînement, et les défis liés au camouflage et aux occultations rendent la tâche de détection plus difficile.

La problématique principale de ce mémoire est donc la suivante :
\textit{Dans quelle mesure le deep learning peut-il améliorer la détection et la reconnaissance en temps réel de véhicules militaires dans des images et vidéos ?}

L’objectif de ce mémoire est d'étudier l'état de l'art des algorithmes de deep learning appliqués à cette problématique, de développer des solutions innovantes pour surmonter les limitations existantes, et de proposer des recommandations pour améliorer la performance de ces systèmes dans des contextes opérationnels.
Pour cela, ce travail intègre l'utilisation de techniques avancées telles que la data augmentation, l'utilisation de modèles génératifs pour enrichir les jeux de données, et l'optimisation des modèles de détection pour les adapter aux contraintes spécifiques du domaine militaire.

Ce projet, nommé \textit{DetReco} (Détection et reconnaissance de véhicules militaires sur des images et vidéos), vise à concevoir une solution technologique robuste, capable de répondre aux besoins opérationnels des acteurs de la défense tout en tenant compte des spécificités du domaine, telles que la confidentialité et la rareté des données disponibles.


\section{Initialisation du projet}
\subsection{Charte du projet}

L’objectif du projet \textit{DetReco} est de développer une solution pour la détection et la reconnaissance en temps réel de véhicules militaires dans des images et des vidéos.
Cette solution repose sur l'optimisation des algorithmes de deep learning afin de surmonter les défis spécifiques rencontrés dans le domaine de la défense, tels que le camouflage, la rareté des données et les environnements complexes.

Pour assurer l'efficacité de cette solution, il est essentiel que les algorithmes et modèles développés évoluent constamment en fonction des nouvelles données et des retours d'expérience provenant des déploiements sur le terrain.
Cela implique que l'équipe de développement doit régulièrement réévaluer les modèles et que toute modification apportée aux structures de données ou à l'environnement d'entraînement doit être suivie d'une mise à jour des algorithmes, garantissant ainsi leur performance opérationnelle.

L’optimisation des modèles de détection permet de réduire les faux positifs et les faux négatifs, augmentant ainsi la précision globale du système.
Elle permet également de s'assurer que les modèles restent performants dans des scénarios variés, en particulier face à des véhicules camouflés ou partiellement occultés.
Cette approche est essentielle pour garantir que la solution réponde aux exigences opérationnelles des acteurs de la défense.

Pour tirer pleinement parti du projet \textit{DetReco}, il est important que les modèles soient régulièrement ajustés et entraîné en fonction des nouvelles situations et images rencontrées lors des recherches.
Ainsi, lorsque de nouvelles images deviennent disponibles, ou lorsque les conditions d'entraînement évoluent, les modèles doivent être réentraînés et les paramètres ajustés en conséquence pour garantir leur pertinence continue.



\begin{table}[h]
    \centering
    % \footnotesize
    \begin{tabular}{|p{4cm}|p{12.5cm}|}
        \hline
        \rowcolor{yellow}\multicolumn{2}{|c|}{\textbf{Agile Project Charter}}                                                                                                                                                                                                                                                                                        \\
        \hline
        \rowcolor{gray}\multicolumn{2}{|c|}{\textbf{General Project Information}}                                                                                                                                                                                                                                                                                    \\
        \hline
        \textbf{Project Name}                        & Détection et reconnaissance de véhicules militaires en temps réel dans des images et vidéos                                                                                                                                                                                                                   \\
        \hline
        \textbf{Project Sponsor}                     & Direction Générale de l'Armement Techniques Terrestres (DGA TT)                                                                                                                                                                                                                                               \\
        \hline
        \textbf{Organization}                        & Inria (Équipe STARS et mission Défense \& Sécurité)                                                                                                                                                                                                                                                           \\
        \hline
        \textbf{Project Start Date}                  & 01/04/2023                                                                                                                                                                                                                                                                                                    \\
        \hline
        \textbf{Project End Date}                    & ----                                                                                                                                                                                                                                                                                                          \\
        \hline
        \rowcolor{gray}\multicolumn{2}{|c|}{\textbf{Project Details}}                                                                                                                                                                                                                                                                                                \\
        \hline
        \textbf{Mission}                             & Le projet vise à développer une solution de détection et de reconnaissance en temps réel de véhicules militaires dans des images et vidéos à l'aide d'algorithmes de deep learning optimisés pour des environnements complexes.                                                                               \\
        \hline
        \textbf{Vision}                              & La solution permettra de différencier les véhicules militaires des autres objets dans des conditions de visibilité réduite ou camouflée, assurant ainsi une meilleure précision pour les opérations militaires.                                                                                               \\
        \hline
        \textbf{Scope}                               & Le projet couvrira l'optimisation des algorithmes de détection d'objets pour les environnements militaires, en mettant l'accent sur les points difficiles tels que faible résolution, faible contraste, occultations et camouflage robustesse face aux variations d'images et la confidentialité des données. \\
        \hline
        \textbf{Success Metrics}                     & Amélioration de la précision de détection à 80\%, réduction des faux positifs de 15\%.                                                                                                                                                                                                                        \\
        \hline
        \textbf{Definition of Done criteria}         & Les modèles de détection auront été intégrés avec succès dans l'environnement opérationnel de test et validés avec des données réelles en conditions militaires.                                                                                                                                              \\
        \hline
        \rowcolor{gray}\multicolumn{2}{|c|}{\textbf{Project Team}}                                                                                                                                                                                                                                                                                                   \\
        \hline
        \textbf{Project Manager}                     & Jonas RENAULT                                                                                                                                                                                                                                                                                                 \\
        \hline
        \textbf{Scrum Master}                        & Jonas RENAULT                                                                                                                                                                                                                                                                                                 \\
        \hline
        \textbf{Technical Lead}                      & Jonas RENAULT                                                                                                                                                                                                                                                                                                 \\
        \hline
        \textbf{Team Members}                        & Équipe STARS, mission Défense \& Sécurité                                                                                                                                                                                                                                                                     \\
        \hline
        \rowcolor{gray}\multicolumn{2}{|c|}{\textbf{Team Rules}}                                                                                                                                                                                                                                                                                                     \\
        \hline
        \textbf{Duration of Sprint}                  & Sprint de deux semaines                                                                                                                                                                                                                                                                                       \\
        \hline
        \textbf{Break Between Sprints}               & > 1 jour                                                                                                                                                                                                                                                                                                      \\
        \hline
        \textbf{Duration of Sprint Planning Meeting} & 1 heure                                                                                                                                                                                                                                                                                                       \\
        \hline
        \textbf{Duration of Daily Scrum Meeting}     & 30 minutes                                                                                                                                                                                                                                                                                                    \\
        \hline
        \textbf{Duration of Sprint Review}           & 30 minutes                                                                                                                                                                                                                                                                                                    \\
        \hline
        \textbf{Duration of Sprint Retrospective}    & 1 heure                                                                                                                                                                                                                                                                                                       \\
        \hline
    \end{tabular}
    \caption{Charte du Projet DetReco}
\end{table}


% \newpage
\clearpage
\subsection{Registre des parties prenantes}

Le registre des parties prenantes est un outil essentiel utilisé dans la gestion de projet pour identifier, analyser et gérer les parties prenantes impliquées dans le projet.
Il permet de recueillir et de consolider les informations clés sur chaque partie prenante afin de mieux comprendre leurs besoins, attentes, intérêts et influences.

Dans le tableau ci-dessous, nous pouvons observer les parties prenantes et leurs caractéristiques.

\begin{table}[H]
    \centering
    % \footnotesize
    \scriptsize
    \begin{tabular}{|p{0.3cm}|p{1.9cm}|p{0.9cm}|p{2.5cm}|p{0.9cm}|p{1cm}|p{1.5cm}|p{2.7cm}|p{2.3cm}|}
        \hline
        \rowcolor{green} \textbf{\#} & \textbf{Nom}           & \textbf{Type} & \textbf{Rôle}                                                        & \textbf{Intérêt} & \textbf{Pouvoir} & \textbf{Stratégie} & \textbf{Contributions}                                                                                          & \textbf{Attentes}                      \\
        \hline
        P1                           & Développeur            & Interne       & Développer les algorithmes de détection et de reconnaissance         & Élevé            & Faible           & Garder Informé     & Ils développent les modèles de deep learning et les optimisent pour les environnements militaires.              & Précision et robustesse des modèles    \\
        \hline
        P2                           & Analyste de données    & Interne       & Préparer et annoter les jeux de données                              & Élevé            & Faible           & Garder Informé     & Ils doivent préparer des jeux de données diversifiés pour entraîner les modèles.                                & Qualité et diversité des données       \\
        \hline
        P3                           & Chercheur              & Interne       & Conduire les recherches nécessaires à l'amélioration des algorithmes & Élevé            & Moyen            & Garder Informé     & Ils mènent des expérimentations pour tester et valider les approches proposées, en apportant des améliorations. & Pertinence et innovation des solutions \\
        \hline
        P4                           & Chef de projet         & Interne       & Manager le projet et l'équipe                                        & Élevé            & Élevé            & Acteur Clé         & Ils coordonnent l'équipe, s'assurent du respect des délais, et communiquent avec les sponsors.                  & Périmètre et délais                    \\
        \hline
        P5                           & Commanditaire (DGA TT) & Externe       & Financer et évaluer le projet                                        & Élevé            & Élevé            & Acteur Clé         & Ils planifient les réunions de suivi, apportent des modifications et valident les livrables.                    & Efficacité des résultats               \\
        \hline
    \end{tabular}
    \caption{Registre des parties prenantes pour le projet DetReco}
\end{table}


% \clearpage

\subsection{Choix de la méthode de gestion du projet}

Pour le projet DetReco, la méthodologie agile Scrum a été choisie pour sa capacité à s'adapter aux projets complexes et flexible.
Scrum divise le travail en sprints, permettant une évaluation régulière des ajustements des modèles et des résultats de leur entraînement en fonction des retours.

Cette méthode est bien adaptée à DetReco, où les algorithmes de deep learning nécessitent des optimisations continues en fonction des nouvelles images et des résultats des entraînements des modèles étudiés.
Elle permet à l'équipe de s'ajuster rapidement aux spécificités des environnements militaires.


\begin{figure}[H]
    % \center
    \includegraphics[width=1\textwidth]{./images/scrum2-0.png}
    \caption[Agile Scrum]{Agile Scrum \cite{exaraw2024}}
    \label{fig:map-train}
\end{figure}




\chapter{Présentation de l'entreprise}
\label{chap:chapterone}
\section{Présentation du centre de recherche}
\label{chap:sectionone}
\sloppy


\subsection{Inria : métiers et chiffres clés.}

L’Inria est l’institut national de recherche en sciences et technologies du numérique, crée en 1967, il dispose de 11 centres et plus de 20 antennes et emploie 2600 personnes. La recherche de rang mondial, l’innovation technologique et le risque entrepreneurial constituent son ADN.
Au sein de 215 équipes-projets en général communes avec des partenaires académiques, plus de 3 900 chercheurs et ingénieurs y explorent des voies nouvelles. 900 personnels d’appui à la recherche et à l’innovation contribuent à faire émerger et grandir des projets scientifiques ou entrepreneuriaux qui impactent le monde.

L’institut fait appel à de nombreux talents dans plus d’une quarantaine de métiers différents, travaille avec de nombreuses entreprises et a accompagné la création de plus de 180 start-ups.
Inria soutient la diversité des voies de l’innovation : de l’édition open source de logiciels à la création de startups technologiques (Deeptech).


\subsection{Le service de l’apprenti : Mission Défense et Sécurité}

Le renforcement des partenariats avec la sphère Sécurité et Défense de l’État est une priorité stratégique de l’Inria. C’est de ce contexte qu’est né le département. Créé en mars 2020 et dirigé par Frédérique Segond, la Mission Défense et Sécurité a pour objectif le soutient des politiques gouvernementales qui visent la souveraineté et l’autonomie stratégique numérique de l’Etat français, voire européen. Elle fédère tous les projets sécurité défense d’Inria bientôt de toute la France.
L’équipe en pleine croissance est actuellement composée de seize personnes ayant chacun un rôle bien définit avec le soutien des intervenants externes à la mission.

\section{Principales activités}


Sous la supervision du responsable informatique, j'ai développé divers outils pédagogiques et technologiques visant à faciliter la prise en main des applications et à enrichir la compréhension des scénarios d'exercice de simulation. Mon travail a contribué à l'élaboration et à l'optimisation de nouvelles technologies au sein de l'équipe.

J'ai principalement participé au développement et à l'évolution de l'application IntelLab qui permet de simuler des scénarios de renseignements militaires et de sécurité économique.
Nous avons utilisé la méthode Agile pour gérer nos projets, ce qui a garanti une bonne organisation et un suivi efficace tout au long du processus.

\noindent Mes contributions se sont réparties sur trois projets distincts :

\begin{itemize}\addtolength{\itemsep}{-0.35\baselineskip}%
  \item \textbf{Développement de l'application web IntelLab }: J'ai pris en charge le développement des interfaces web Frontend et Backend de la plateforme, intégrant les dernières technologies pour garantir une performance et une expérience utilisateur optimales.
  \item \textbf{Mise en place d'un serveur de messagerie} : En parallèle, j'ai contribué au déploiement d'une plateforme dédiée à la simulation de scénarios de sécurité économique, centrée sur la gestion d'un serveur de messagerie. Cette plateforme est utilisée dans des exercices de simulation reproduisant des environnements d'entreprises tels que des startups ou des équipes de recherche.
  \item \textbf{Détection et reconnaissance, proche du temps réel, de véhicules militaires sur des images et vidéos} : J'ai participé à un projet de développement et d'amélioration d'algorithmes de deep learning pour la détection et la reconnaissance en temps réel de véhicules militaires dans des images et vidéos.
\end{itemize}

\noindent Dans le cadre de ces projets, mes responsabilités incluent :

\begin{itemize}\addtolength{\itemsep}{-0.35\baselineskip}%
  \item \textbf{Développement web Frontend et Backend} : Conception et implémentation des fonctionnalités sur les deux volets de l'application.
  \item \textbf{Développement d'algorithmes génératifs} : Création et intégration d'algorithmes destinés à enrichir les jeux de données pour améliorer les performances des modèles.
  \item \textbf{Rédaction des tests} : Rédaction et exécution de tests pour garantir la fiabilité et la robustesse des solutions développées.
  \item \textbf{Mise en production} : Déploiement des solutions dans l'environnement de production, assurant leur bon fonctionnement et leur maintenance.
  \item \textbf{Maintenance des applications et de leurs infrastructures} : Assurer la stabilité et la disponibilité continue des applications en identifiant et résolvant les problèmes techniques rapidement.
\end{itemize}



\chapter{Modèles utilisés et Applications}
\label{chap:2}
\sloppy

Les ambitions de la mission Défense et Sécurité sont aujourd’hui implémentées à travers un centre d’excellence dédié au domaine de la Sécurité et Défense afin de faciliter le développement et le transfert à court, moyen et long terme de technologies issues de la Recherche.

Ce chapitre est celui dans lequel nous allons présenter les projets sur les quels nous avons travailler, notre méthodologie de travail, leur analyse et les résultats obtenus durant notre alternance au seins de l'Inria.

\section{Projet IntelLab : Application Web}

\subsection{Contexte du projet}

IntelLab est un environnement de \textbf{simulation, formation, et d’expérimentation}. Il vise d’une part à faire appréhender aux académiques et aux entreprises les problèmes concrets rencontrés par les opérationnels afin d’y proposer des solutions communes, et d’autre part, de permettre d’expérimenter les solutions sur la base de procédures de tests opérationnels.

\noindent Les plateformes du projet IntelLab permettent de jouer deux types de scénarios :

\begin{itemize}\addtolength{\itemsep}{-0.35\baselineskip}%
	\item Des scénarios simulant l'exploitation du renseignement d'intérêt militaire ;
	\item Des scénarios de type sécurité économique (détection de signaux faibles précurseurs d'actions d'ingérence).
\end{itemize}

\subsection{Moyens mis à la disposition des équipes de développement}
Dans le cadre de ce projet, l'équipe de développement est composée uniquement du responsable informatique et de moi-même. Cela signifie que nous avons adopté une approche \textit{agile et collaborative} pour gérer le projet.


\subsection{Outils de gestion de projet}

Les projets sont suivis lors de réunions hebdomadaires et de séminaires semestriels, où chaque membre de l'équipe évalue les progrès, identifie les problèmes et propose des corrections pour atteindre les objectifs fixés.
En complément, des échanges bilatéraux hebdomadaires entre responsables et membres de l'équipe permettent de prendre des décisions plus rapidement et efficacement, en se concentrant sur des discussions plus ciblées que lors des réunions d'équipe générales.

\subsubsection{Définition des objectifs et des exigences du projet}

Il était important de recueillir les exigences et attentes de l’application.
Grâce à cela nous avons planifié notre travail en fonction des priorités et urgences. Cette planification a été faites sur GitLab sous forme d’issues différenciés par des labels (backend, frontend, bug, …).
Ces labels nous permettent de déterminer dans quoi et ou à quelle partie correspondes les issues.

\subsubsection{Planification des tâches}

Pour le projet IntelLab, notre méthode de travail est fortement axée vers la méthodologie agile car les priorités du développement varient régulièrement en fonction du calendrier des scénarios joués ou à jouer sur les applications.
Le responsable informatique Jonas Renault étant chargé de la planification, a choisi d'utiliser la plateforme de gestion de code source GitLab.

\begin{figure}[h]
	\center
	\includegraphics[width=\textwidth]{./images/gitlab_intellab.PNG}
	\caption[Planification des issues]{Planning issues - GitLab}\label{fig:gitlab_intellab}
\end{figure}

GitLab est une plateforme de gestion de code source basée sur Git qui permet aux équipes de développeurs de collaborer sur des projets de logiciels, de suivre les modifications du code source et de gérer des versions de code.
Elle offre des fonctionnalités pour la collaboration en temps réel, l'intégration continue et la livraison continue, la gestion de projet.


\subsection{Réalisation du projet}
IntelLab anciennement appelé \textbf{BLR (Battle Lab Rens)} a été initialement développé par un ancien ingénieur. Le BLR avait pour objectif la simulation exclusive des scénarios des services de renseignements militaires.
Dans l’optique de rendre l’application plus générique afin qu’elle s’adapte à d’autres types de scénarios autres que le renseignement militaire, le responsable informatique et moi avions procédé à la refonte complète de l’application en commençant par le changement de nom.


\subsubsection{Analyse du code existant et développement de la plateforme}

Avant de commencer le développement de l’application, nous avons effectué une analyse approfondie du code et de l’infrastructure du projet.
Ce qui nous a permis de mettre en place une logique de travail sur la restructuration de tout le projet.

Nous avons fait une refonte complète de l’application. La refonte avait pour but de restructurer le code afin de pouvoir poursuivre le développement de fonctionnalités nouvelles et de nouvelles fonctionnalités.
En parallèle, nous avons rédigé des tests unitaires et fonctionnelle pour garantir l'absence de détérioration dans la logique fonctionnelle de l'application après avoir effectué le remaniement du code que nous avons orchestré.


\subsubsection{Infrastructure des serveurs de développement et de production}

Il n'existait pas de serveur de développement sur lequel nous pouvions effectuer des tests de déploiement. Nous avons mis en place un serveur de développement qui sera par la suite la réplique parfaite du serveur de production.
Nous avons choisi d'utiliser Docker comme environnement d'exécution, installé sur un système d'exploitation Ubuntu Server.
La conteneurisation de nos serveurs nous offre une grande flexibilité dans la gestion des composants de notre application.
Ce serveur sert à vérifier le bon fonctionnement de l'application avant sa mise en production.

Les serveurs du backend, du frontend, du broker MQTT et de la base de données sont automatiquement installés dans l'environnement Docker des serveurs de développement et de production.
Cela est possible grâce au déploiement continu depuis GitLab que nous avons configuré, ce qui nous fait gagner énormément de temps pendant le déploiement.


\subsubsection{Description de l'infrastructure logicielle}

Cet environnement est développé en utilisant les technologies suivantes :

\begin{itemize}
	\item \textbf{Frontend} : ReactJS, gère l’interface utilisateur ;
	\item \textbf{Backend} :
	      \begin{itemize}
		      \item \textit{NodeJS} et \textit{ExpressJS} pour la partie serveur web ;
		      \item \textit{PostgreSQL} est le système de gestion de base de données ;
		      \item \textit{Sequelize} est utilisé pour les requêtes entre le serveur et la base de données ;
		      \item \textit{MQTT} : Il fournit une méthode de communication asynchrone de messages entre deux ou plusieurs appareils connectés à un réseau.
	      \end{itemize}
	\item \textbf{Infrastructure} :
	      \begin{itemize}
		      \item \textit{Docker} : Utilisé pour le déploiement de l’application sur les serveurs de test et de production ;
		      \item \textit{GitLab} : notre projet y est répertorié pour le travail collaboratif, les tests unitaires, les tests d’intégration, déploiement et intégration automatique.
	      \end{itemize}
	\item \textbf{Tests} :
	      \begin{itemize}
		      \item \textit{Vitest} : Outils de gestion des tests côté frontend ;
		      \item \textit{Jest} : Outils de gestion des tests côté backend.
	      \end{itemize}
\end{itemize}


\begin{figure}[h]
	\center%
	\includegraphics[width=\textwidth]{./images/architectur_intellab.png}
	\caption[Planification des issues]{Planning issues - GitLab}\label{fig:architectur_intellab}
\end{figure}


\subsection{Stratégie de tests}

En tant qu'environnement de simulation, formation et expérimentation, la fiabilité et la qualité d’IntelLab sont d'une importance capitale pour atteindre les objectifs visés par cette plateforme.
IntelLab est une application de recherche, donc nous n’avons pas les mêmes attentes ni besoins en termes de stratégie de tests.


\subsubsection{Objectifs et critères de tests}
Les tests sur IntelLab visent à assurer que l'application répond aux exigences fonctionnelles, en permettant des simulations réalistes et fiables des scénarios de renseignement.
Ils visent également à garantir la stabilité et la sécurité des données, à détecter et corriger les anomalies pendant les exercices, et à améliorer l'ergonomie et la convivialité de l'application pour une meilleure expérience utilisateur.


\subsubsection{Mise en place des outils de tests}

\begin{itemize}\addtolength{\itemsep}{-0.35\baselineskip}%
	\item \textbf{Outils de gestion des tests }: Nous utilisons l'outil de gestion des tests \textbf{vitest} coté frontend et \textbf{jest} coté backend pour suivre nos cas de test, les résultats des tests et les anomalies détectées.
	      \begin{figure}[h]
		      \center
		      \includegraphics[scale=0.7]{./images/test_frontend_il.png}
		      \caption[Rapport test frontend avec VITEST]{Rapport tests avec VITEST}\label{fig:test_frontend_il}
	      \end{figure}
	\item \textbf{Outils d'automatisation des test} : Nous utilisons les pipelines de GitLab pour automatiser l'exécution des tests.
	      \begin{figure}[h]
		      \center
		      \includegraphics[scale=0.9]{./images/pipeline_tests.png}
		      \caption[pipeline des tests automatiques]{pipeline des tests automatiques}\label{fig:pipeline_tests}
	      \end{figure}
\end{itemize}


\subsubsection{Rédaction des procédures de tests}

\begin{itemize}
	\item \textbf{Scénarios de test} : Chaque scénario est accompagné de données de test spécifiques et de critères de succès clairement définis.
	\item \textbf{Procédures de test} : Nous documentons les procédures de test pour chaque type de test afin de garantir une exécution cohérente.
	\item \textbf{Critères d'acceptation} : Nous établissons des critères d'acceptation clairs pour chaque scénario de test, définissant ce qui est considéré comme un test réussi.
\end{itemize}

\begin{table}[h]
	\centering
	\begin{tabular}{|p{1.1cm}|p{3cm}|p{3cm}|p{4cm}|p{3cm}|}
		\hline
		\textbf{Menu Name} & \textbf{Description}                & \textbf{Test data}                & \textbf{Expected Output}                                     & \textbf{Actual Output}                                       \\ \hline
		Login              & should render a sign in button      & -                                 & Se connecter                                                 & Se connecter                                                 \\ \hline
		Login              & should display required helper text & Username: empty Password: Empty   & Le nom d'utilisateur est requis. Un mot de passe est requis. & Le nom d'utilisateur est requis. Un mot de passe est requis. \\ \hline
		Login              & should login user                   & Username: john Password: johnjohn & Connexion réussie                                            & Connexion réussie                                            \\ \hline
		Login              & should handle server errors         & Username: john Password: johnpass & Incorrect email or password                                  & Incorrect email or password                                  \\ \hline
	\end{tabular}
	\caption{Scénarios de test}
	\label{tab:test_scenarios}
\end{table}


\section{Projet IntelLab : Serveur de messagerie}

Ce projet a beaucoup de similitudes que celui de l'application Web. Nous allons ressortir quelques paries avec leur différences :


\subsection{Contexte du projet}
Le contexte initial de ce projet est pratiquement le même que celui de l'application web, la différence réside au niveau du type de scénario.
Ce projet permet la simulation des scénarios de types sécurité économique.
Ce type de scénario a exigé le développement d'une autre d'environnement car diffère du monde du renseignement militaire.
Cette plateforme est né du fait qu'on doit recréer un environnement d’entreprises telles que des startups ou des équipes de recherche.
Les objectifs la formation et la sensibilisation à la détection de signaux faibles précurseurs d’actions d’ingérence.


\subsection{Outils de gestion de projet}
Comme pour les autres projets, nous sommes une équipe de deux personnes, le responsable informatique et moi. Au vu de cela, nous avons toujours adopté la méthodologie agile car cela nous permet de gérer efficacement nos ressources limitées, de nous adapter rapidement aux changements de priorités, et de livrer des fonctionnalités clés du serveur de messagerie de manière itérative et continue, en garantissant une meilleure réactivité aux besoins des utilisateurs finaux.


\subsubsection{Planification des tâches}
Comme pour l'autre projet, nous avons centralisé la gestion de ce projet sur la plateforme GitLab.
La description des taches, la gestion du code, collaboration en temps réel, l’intégration continue et le déploiement automatique, toutes ses fonctionnalités sont gérées sur la plateforme GitLab.


\begin{figure}[h]
	\center
	\includegraphics[width=0.6\textwidth]{./images/gitlab_mpc.png}
	\caption[Planification des issues MPC]{Planning issues MPC - GitLab}\label{fig:gitlab_mpc}
\end{figure}





































% --------------------------------------------------------------------------
\section{Projet Détection et reconnaissance de véhicules militaires sur des images et vidéos (DetReco)}

\subsection{Context du projet DetReco}

Le projet DetReco s'inscrit dans une collaboration stratégique entre l'équipe STARS du centre de Sophia-Antipolis, la Direction Générale de l'Armement (DGA), et le département Défense et Sécurité de l'Inria. Ce projet a pour objectif de mener une étude approfondie de l'état de l'art des algorithmes appliqués à la détection et à la reconnaissance, en temps quasi réel, de véhicules militaires sur des images et vidéos.



\section{Analyse}

\section{Résultats obtenus}







% \subsection{Inria : métiers et chiffres clés.}



% \begin{itemize}
% 	\item The individual \index{Entries}{entries} are indicated with a black dot, a so-called bullet.
% 	\item The text in the entries may be of any length.
% \end{itemize}

% \begin{theorem}\label{theo1}
% 	Soit $n$ un entier naturel. Si $n$ est premier alors il n'est divisible que par 1 et par lui-même.
% \end{theorem}

% \begin{proof}
% 	Here is my proof.
% \end{proof}

% \begin{definition}\label{def1}
% 	Soit $A$ une courbe...
% \end{definition}

% Ici, il s'agit de l'utilisation de TB %\nomenclature[TB]{TB}{Très Bien} qui consiste à parler Très Bien. 
% \gls{abc} et \gls{efg} sont des acronyms et des abbréviations... La méthode \gls{svm} est également couramment utilisée.

% \begin{exemple}\label{exo1}
% 	On considère le cas particulier...
% \end{exemple}


\chapter*{Conclusion et Perspectives}
\label{chap:conclusion}
\markboth{\MakeUppercase{Conclusion}}{}%
\addcontentsline{toc}{chapter}{Conclusion}
And a very interesting conclusion here\@. ~\\
Lorem ipsum dolor sit amet, consectetur adipisicing elit, sed do eiusmod
tempor incididunt ut labore et dolore magna aliqua. Ut enim ad minim veniam,
quis nostrud exercitation ullamco laboris nisi ut aliquip ex ea commodo
consequat.

\newpage
\appendix
\addcontentsline{toc}{chapter}{Annexes}
%\markboth{\MakeUppercase{Annexe}}{}

\chapter{Code R pour résoudre la problématique}
\label{chap:appendix}


\section{Pré-traitement des données}
\section{Code R pour les modèles}

An appedix if you need it.

\begin{verbatim}
 Insérer ici le code !
 \end{verbatim}

\section{Librairies utilisées}

Lorem ipsum dolor sit amet, consectetur adipisicing elit, sed do eiusmod
tempor incididunt ut labore et dolore magna aliqua. Ut enim ad minim veniam,
quis nostrud exercitation ullamco laboris nisi ut aliquip ex ea commodo.


%%%%%%%%%%%%%%%%%%%%%%%%%%%%%%%%%%%%%%%%%%%%%%%%%%%%
% Don't touch this, it is auto generated
%%%%%%%%%%%%%%%%%%%%%%%%%%%%%%%%%%%%%%%%%%%%%%%%%%%%
\nocite{*}

%\phantomsection{}
%\addcontentsline{toc}{chapter}{Webography}
%\printbibliography[title={Webography},type=online]

%\phantomsection{}
%\addcontentsline{toc}{chapter}{Bibliography}
%\printbibliography[title={Bibliography},nottype=online]

%\printbibheading %exemple de bibliographie divisée en sections. Pour ajouter des oeuvres non citées,utiliser \nocite

%\printbibliography[keyword=pratique,heading=subbibliography,title={Théories littéraires dans les jeux vidéo}]
%\printbibliography[keyword=litteraire,heading=subbibliography,title={Narratologie et structuralisme}]

%\printbibliography[keyword=jeu,heading=subbibliography,title={\emph{Games studies}}]

\bibliographystyle{apalike}
%\bibliographystyle{plain}

\bibliography{Biblio.bib}

\cleardoublepage%

\addtocontents{toc}{\protect\setcounter{tocdepth}{3}}

\printglossaries
\printindex

\markboth{\MakeUppercase{Abstract}}{}
\addcontentsline{toc}{chapter}{Abstract}
\sloppy

The present work is part of a graduation project carried out within the company  in order to obtain the national diploma of engineer at the \IMIE. This project's objective is to design and implement a ... \\




\end{document}
