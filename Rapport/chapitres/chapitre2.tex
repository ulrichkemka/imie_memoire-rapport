\sloppy

% Les ambitions de la mission Défense et Sécurité sont aujourd’hui implémentées à travers un centre d’excellence dédié au domaine de la Sécurité et Défense afin de faciliter le développement et le transfert à court, moyen et long terme de technologies issues de la Recherche.

Ce chapitre est celui dans lequel nous allons présenter les projets sur lesquels nous avons travaillé, notre méthodologie de travail, leur analyse et les résultats obtenus durant notre alternance au sein d'inria.

\section{Projet IntelLab : Application Web}

\subsection{Contexte du projet}

IntelLab est un environnement de \textbf{simulation, formation, et d’expérimentation}. Il vise d’une part à faire appréhender aux académiques et aux entreprises les problèmes concrets rencontrés par les opérationnels afin d’y proposer des solutions communes, et d’autre part, de permettre d’expérimenter les solutions sur la base de procédures de tests opérationnels.

\noindent Les plateformes du projet IntelLab permettent de jouer deux types de scénarios :

\begin{itemize}\addtolength{\itemsep}{-0.35\baselineskip}%
	\item Des scénarios simulant l'exploitation du renseignement d'intérêt militaire ;
	\item Des scénarios de type sécurité économique (détection de signaux faibles précurseurs d'actions d'ingérence).
\end{itemize}

\subsection{Ressources disponibles}
Dans le cadre de ce projet, l'équipe de développement est composée uniquement du responsable informatique et de moi-même. Nous avons adopté une approche \textit{agile et collaborative} pour gérer le projet.
Cette méthode nous a permis de maximiser notre efficacité en encourageant une communication régulière et une adaptation rapide aux besoins et évolutions du projet.


\subsection{Outils de gestion de projet}

Les projets sont suivis lors de réunions hebdomadaires et de séminaires semestriels, où chaque membre de l'équipe évalue les progrès, identifie les problèmes et propose des corrections pour atteindre les objectifs fixés.
En complément, des échanges bilatéraux hebdomadaires entre responsables et membres de l'équipe permettent de prendre des décisions plus rapidement et efficacement, en se concentrant sur des discussions plus ciblées que lors des réunions d'équipe générales.

\subsubsection{Définition des objectifs et des exigences du projet}

Il était important de recueillir les exigences et attentes de l’application.
Grâce à cela nous avons planifié notre travail en fonction des priorités et urgences. Cette planification a été faites sur GitLab sous forme d’issues différenciés par des labels (backend, frontend, bug, …).
Ces labels nous permettent de catégoriser et d'organiser les issues.

\subsubsection{Planification des tâches}

Pour le projet IntelLab, notre méthode de travail est fortement axée vers la méthodologie agile car les priorités du développement varient régulièrement en fonction du calendrier des scénarios joués ou à jouer sur les applications.
Le responsable informatique Jonas Renault étant chargé de la planification, a choisi d'utiliser la plateforme de gestion de code source GitLab.

\begin{figure}[h]
	\center
	\includegraphics[width=\textwidth]{./images/gitlab_intellab.PNG}
	\caption[Planification des issues]{Planification des tâches - GitLab}\label{fig:gitlab_intellab}
\end{figure}

GitLab est une plateforme de gestion de code source basée sur Git qui permet aux équipes de développeurs de collaborer sur des projets de logiciels, de suivre les modifications du code source et de gérer des versions de code.
Elle offre des fonctionnalités pour la collaboration en temps réel, l'intégration continue et la livraison continue, la gestion de projet.


\subsection{Réalisation du projet}
IntelLab anciennement appelé \textbf{BLR (Battle Lab Rens)} a été initialement développé par un ancien ingénieur. Le BLR avait pour objectif la simulation exclusive des scénarios des services de renseignements militaires.
Dans l’optique de rendre l’application plus générique afin qu’elle s’adapte à d’autres types de scénarios autres que le renseignement militaire, le responsable informatique et moi avions procédé à la refonte complète de l’application en commençant par le changement de nom.


\subsubsection{Analyse du code existant et développement de la plateforme}

Avant de commencer le développement de l’application, nous avons effectué une analyse approfondie du code et de l’infrastructure du projet.
Ce qui nous a permis de mettre en place une logique de travail sur la restructuration de tout le projet.

Nous avons fait une refonte complète de l'application pour répondre à plusieurs objectifs : simplifier le code, améliorer la performance, et intégrer de nouvelles fonctionnalités demandées par les utilisateurs.\\

\noindent Les travaux réalisés sont:

\begin{itemize}
	\item la suppression des parties de code redondantes ou obsolètes.
	\item la restructuration des API REST en utilisant des méthodes appropriées comme \texttt{GET}, \texttt{PUT}, \texttt{DELETE} et \texttt{POST} au lieu de les regrouper sous une seule méthode \texttt{POST}.
	\item l'isolation de chaque composant pour qu'il soit responsable d'une tâche précise, ce qui a réduit la complexité et amélioré la maintenabilité du code.
	\item l'utilisation des hooks, des composants en fonction au lien des classes, pour tirer parti des fonctionnalités récentes de ReactJS.
	\item la refonte de la navigation pour la rendre plus intuitive en introduisant des éléments interactifs plus clairs et des raccourcis qui facilitent la navigation, réduisant ainsi le temps d'adaptation des nouveaux utilisateurs.
\end{itemize}

Cette refonte a permis non seulement de rendre le code plus propre et plus facile à maintenir, mais aussi d'améliorer significativement la performance de l'application et l'expérience utilisateur.


\subsubsection{Infrastructure des serveurs de développement et de production}

Il n'existait pas de serveur de développement sur lequel nous pouvions effectuer des tests de déploiement. Nous avons mis en place un serveur de développement qui sera par la suite la réplique parfaite du serveur de production.
Nous avons choisi d'utiliser Docker comme environnement d'exécution, installé sur un système d'exploitation Ubuntu Server.
La conteneurisation de nos serveurs nous offre une grande flexibilité dans la gestion des composants de notre application.
Ce serveur sert à vérifier le bon fonctionnement de l'application avant sa mise en production.

Les containers du backend, du frontend, du broker MQTT et de la base de données sont automatiquement installés dans l'environnement Docker des serveurs de développement et de production.
Cela est possible grâce au déploiement continu depuis GitLab que nous avons configuré, ce qui nous fait gagner énormément de temps pendant le déploiement.


\subsubsection{Description de l'infrastructure logicielle}

Cet environnement est développé en utilisant les technologies suivantes :

\begin{itemize}
	\item \textbf{Frontend} : ReactJS, gère l’interface utilisateur ;
	\item \textbf{Backend} :
	      \begin{itemize}
		      \item \textit{NodeJS} et \textit{ExpressJS} pour la partie serveur web ;
		      \item \textit{PostgreSQL} est le système de gestion de base de données ;
		      \item \textit{Sequelize} est utilisé pour les requêtes entre le serveur et la base de données ;
		      \item \textit{MQTT} : Il fournit une méthode de communication asynchrone de messages entre deux ou plusieurs appareils connectés à un réseau.
	      \end{itemize}
	\item \textbf{Infrastructure} :
	      \begin{itemize}
		      \item \textit{Docker} : Utilisé pour le déploiement de l’application sur les serveurs de test et de production ;
		      \item \textit{GitLab} : notre projet y est répertorié pour le travail collaboratif, les tests unitaires, les tests d’intégration, déploiement et intégration automatique.
	      \end{itemize}
	\item \textbf{Tests} :
	      \begin{itemize}
		      \item \textit{Vitest} : Outils de gestion des tests côté frontend ;
		      \item \textit{Jest} : Outils de gestion des tests côté backend.
	      \end{itemize}
\end{itemize}


\begin{figure}[h]
	\center%
	\includegraphics[width=\textwidth]{./images/architectur_intellab.png}
	\caption[Architecture Application Web IntelLab]{Architecture Application Web IntelLab}\label{fig:architectur_intellab}
\end{figure}


\subsection{Stratégie de tests}

En tant qu'environnement de simulation, formation et expérimentation, la fiabilité et la qualité d’IntelLab sont d'une importance capitale pour atteindre les objectifs visés par cette plateforme.
IntelLab est une application de recherche, donc nous n’avons pas les mêmes attentes ni besoins en termes de stratégie de tests.


\subsubsection{Objectifs}
Les tests sur IntelLab visent à assurer que l'application répond aux exigences fonctionnelles, en permettant des simulations réalistes et fiables des scénarios de renseignement.
Ils visent également à garantir la stabilité et la sécurité des données, à détecter et corriger les anomalies pendant les exercices, et à améliorer l'ergonomie et la convivialité de l'application pour une meilleure expérience utilisateur.


\subsubsection{Mise en place des outils de tests}

\begin{itemize}\addtolength{\itemsep}{-0.35\baselineskip}%
	\item \textbf{Outils de gestion des tests }: Nous utilisons l'outil de gestion des tests \textbf{vitest} coté frontend et \textbf{jest} coté backend pour suivre nos cas de test, les résultats des tests et les anomalies détectées.
	      \begin{figure}[h]
		      \center
		      \includegraphics[scale=0.5]{./images/test_frontend_il.png}
		      \caption[Rapport test frontend avec VITEST]{Rapport tests avec VITEST}\label{fig:test_frontend_il}
	      \end{figure}
	\item \textbf{Outils d'automatisation des test} : Nous utilisons les pipelines de GitLab pour automatiser l'exécution des tests.
	      \begin{figure}[h]
		      \center
		      \includegraphics[scale=0.8]{./images/pipeline_tests.png}
		      \caption[pipeline des tests automatiques]{pipeline des tests automatiques}\label{fig:pipeline_tests}
	      \end{figure}
\end{itemize}


\subsubsection{Rédaction des procédures de tests}

\begin{itemize}
	\item \textbf{Scénarios de test} : Chaque scénario est accompagné de données de test spécifiques et de critères de succès clairement définis.
	\item \textbf{Procédures de test} : Nous documentons les procédures de test pour chaque type de test afin de garantir une exécution cohérente.
	\item \textbf{Critères d'acceptation} : Nous établissons des critères d'acceptation clairs pour chaque scénario de test, définissant ce qui est considéré comme un test réussi.
\end{itemize}

\begin{table}[H]
	\centering
	\begin{tabular}{|p{1.1cm}|p{3cm}|p{3cm}|p{4cm}|p{3cm}|}
		\hline
		\textbf{Menu Name} & \textbf{Description}                & \textbf{Test data}                & \textbf{Expected Output}                                     & \textbf{Actual Output}                                       \\ \hline
		Login              & should render a sign in button      & -                                 & Se connecter                                                 & Se connecter                                                 \\ \hline
		Login              & should display required helper text & Username: empty Password: Empty   & Le nom d'utilisateur est requis. Un mot de passe est requis. & Le nom d'utilisateur est requis. Un mot de passe est requis. \\ \hline
		Login              & should login user                   & Username: john Password: johnjohn & Connexion réussie                                            & Connexion réussie                                            \\ \hline
		Login              & should handle server errors         & Username: john Password: johnpass & Incorrect email or password                                  & Incorrect email or password                                  \\ \hline
	\end{tabular}
	\caption{Scénarios de test}
	\label{tab:test_scenarios}
\end{table}


\section{Projet IntelLab : Serveur de messagerie}

Ce projet a beaucoup de similitudes que celui de l'application Web. Nous allons ressortir quelques parties avec leur différences :


\subsection{Contexte du projet}
Le contexte initial de ce projet est pratiquement le même que celui de l'application web, la différence réside au niveau du type de scénario.
Ce projet permet la simulation des scénarios de types sécurité économique.
Ce type de scénario a exigé le développement d'un autre d'environnement car diffère du monde du renseignement militaire.
Cette plateforme est née du fait qu'on doit recréer un environnement d’entreprise telles que des startups ou des équipes de recherche.
Les objectifs sont la formation et la sensibilisation à la détection de signaux faibles précurseurs d’actions d’ingérence.

Pour ce faire, nous avons configuré un serveur de messagerie simulant des échanges électroniques entre des utilisateurs fictifs au sein de l’environnement.
Des scripts ont été développés pour initialiser les boîtes de réception, effectuer des sauvegardes, et automatiser l’envoi de messages.
Ce dispositif permet de reproduire fidèlement les interactions courantes en entreprise, afin de sensibiliser les utilisateurs à la détection de menaces dans un cadre réaliste.

\subsection{Outils de gestion de projet}
Comme pour les autres projets, nous sommes une équipe de deux personnes, le responsable informatique et moi. Au vu de cela, nous avons toujours adopté la méthodologie agile car cela nous permet de gérer efficacement nos ressources limitées, de nous adapter rapidement aux changements de priorités, et de livrer des fonctionnalités clés du serveur de messagerie de manière itérative et continue, en garantissant une meilleure réactivité aux besoins des utilisateurs finaux.

\subsubsection{Planification des tâches}
Comme pour l'autre projet, nous avons centralisé la gestion de ce projet sur la plateforme GitLab.
La description des taches, la gestion du code, collaboration en temps réel, l’intégration continue et le déploiement automatique, toutes ces fonctionnalités sont gérées sur la plateforme GitLab.


\subsection{Réalisation du projet}

\subsubsection{Etude de l'existant}

Dans le cadre du projet de serveur de messagerie, nous avons étudié l'infrastructure existante.
Cette infrastructure présente un serveur de messagerie utilisant \textit{HmailServer}, avec \textit{Mozilla Thunderbird} et \textit{Outlook} pour les utilisateurs finaux.

L'ancienne infrastructure reposait sur l'installation d'HmailServer sur un PC dédié en local, non utilisé par les joueurs, et nécessitait la configuration de répéteurs WiFi pour permettre la connectivité des utilisateurs au réseau de messagerie.
Les adresses IP et les noms de domaine devaient être manuellement configurés pour garantir la connexion des PC clients au PC contenant le serveur de messagerie.
La gestion des comptes de messagerie était effectuée via HmailServer, où chaque domaine et adresse étaient créés manuellement avant d'être ajoutés dans Mozilla Thunderbird et ou Outlook.

Cette approche offrait une solution locale robuste, mais nécessitait une configuration minutieuse et fastidieuse, notamment lors de changements d'adresses IP ou de la configuration des serveurs SMTP dans Thunderbird.
Nous avons également souligné des limitations, telles que la nécessité d'ouvrir certains ports spécifiques pour que le serveur puisse fonctionner correctement sur certains PC, ce qui impliquait l'intervention de la DSI.

L’analyse de l'infrastructure actuelle a mis en lumière les points forts et les limites de l’ancien système, fournissant ainsi des indications nécessaires pour améliorer et actualiser le nouveau projet de serveur de messagerie.


\subsubsection{Nouvelle infrastructure du serveur de messagerie}

Le projet de serveur de messagerie utilise une infrastructure basée sur Docker pour les environnements de développement et de production. Docker permet de conteneuriser les différents services nécessaires au fonctionnement du serveur de messagerie, offrant ainsi une grande flexibilité et une gestion simplifiée des déploiements.
Cette approche permet également de maintenir des environnements cohérents entre le développement et la production, assurant une transition fluide et sans heurts.

Les services critiques, tels que le serveur SMTP, IMAP, l'interface d'administration, ainsi que les services auxiliaires comme le résolveur DNS, sont déployés dans des conteneurs Docker distincts. Cette architecture modulaire permet une maintenance facilitée et un déploiement rapide des mises à jour.

\subsubsection{Description de l'infrastructure}

\textbf{Mailu} est un serveur de messagerie simple mais complet, facile à configurer et à entretenir.

Cet environnement est développé en utilisant un ensemble d'images docker utilisées pour configurer et gérer le serveur de messagerie.\\

\noindent Les services principaux incluent :
\begin{itemize}
	\item \textbf{Serveur de messagerie standard,} IMAP et IMAP+, SMTP et soumission avec profils de configuration automatique pour les clients.
	\item \textbf{Fonctionnalités de messagerie avancées,} alias, alias de domaine, routage personnalisé, recherche en texte intégral des pièces jointes des e-mails
	\item \textbf{Accès Web (Roundcube),} fournit les services IMAP pour l'accès aux boîtes aux lettres.
	\item \textbf{Fonctionnalités utilisateur,} alias, réponse automatique, transfert automatique, comptes récupérés, managesieve.
	\item \textbf{Fonctionnalités d'administration,} un service de filtrage de spam qui analyse et marque les emails suspects.
	\item \textbf{Sécurité,} TLS appliqué, DANE, MTA-STS, Letsencrypt!, DKIM sortant, scanner antivirus, Snuffleupagus , blocage des pièces jointes malveillantes.
	\item \textbf{Antispam,} auto-apprentissage, liste grise, DMARC et SPF, anti-usurpation d'identité.
	\item \textbf{Transparence,} aucun tracker inclus dans tous les composants.
\end{itemize}


\noindent La mise en place de l'infrastructure a nécessité l'utilisation de :
\begin{itemize}
	\item \textbf{Docker :}  pour déployer et orchestrer l'ensemble des services de messagerie sur les serveurs de développement et de production.
	\item \textbf{Poetry :}  pour gérer les dépendances Python et les scripts d'administration du serveur. Voir la figure \ref{fig:docker_mpc}
\end{itemize}

\begin{figure}[h]
	\center
	\includegraphics[width=0.6\textwidth]{./images/docker_mpc.png}
	\caption[Liste des conteneurs de Mailu]{Containers Mailu}\label{fig:docker_mpc}
\end{figure}


\subsubsection{Scripts d'administration}

L'objectif principal de ce projet était d'automatiser et de simplifier la mise en place, le déploiement, et l'administration des boîtes mails utilisées dans le cadre des scénarios de simulation.

Pour cela, j'ai développé des \textbf{scripts en Python et Bash} qui permettent de :

\begin{itemize}
	\item Configurer les comptes utilisateurs sur le serveur de messagerie.
	\item Initialiser les boîtes aux lettres des utilisateurs avec les mails nécessaires pour le déroulement des scénarios.
	\item Sauvegarder les mails des utilisateurs à la fin de chaque session de jeu.
	\item Réinitialiser et reconfigurer le serveur de messagerie pour chaque nouvelle session.
	\item Restaurer les mails d’une session antérieure pour des sessions ultérieures, selon les besoins des scénarios.
\end{itemize}

La rédaction de ces scripts est ma principale contribution au projet.
Ils ont été spécifiquement conçus pour répondre aux exigences du projet, en assurant une administration fluide et efficace des boîtes mails au sein de l'environnement de simulation.


\subsubsection{Avantages}

La nouvelle infrastructure présente plusieurs avantages par rapport à l'ancienne :

\begin{itemize}
	\item \textbf{Flexibilité et Scalabilité :} L'utilisation de Docker pour conteneuriser les différents services permet une grande flexibilité dans la gestion des composants du système.
	      Cette approche facilite le déploiement, la mise à jour, et la maintenance des services, tout en permettant de faire évoluer l'infrastructure selon les besoins sans l'interruption des opérations en cours.
	\item \textbf{Automatisation et Gestion Simplifiée} : L'intégration des scripts d'administration, que j'ai développés en \textbf{Python et Bash}, a considérablement simplifié la gestion des comptes utilisateurs, l'initialisation des boîtes mails, la sauvegarde, la réinitialisation et la restauration des données sur le serveur de messagerie.
	      Ces scripts automatisent des tâches importante, telles que la création des comptes et la gestion des boîtes mails, qui auparavant auraient nécessité une intervention manuelle complexe et fastidieuse.
	      Grâce à ces scripts, l'administration du serveur de messagerie est plus rapide et plus fiable, car toutes opérations sont standardisés et automatisés.
	      Cela réduit les erreurs humaines et améliore l'efficacité opérationnelle du l'infrastructure.
\end{itemize}




% --------------------------------------------------------------------------
\section{Projet Détection et reconnaissance de véhicules militaires sur des images et vidéos (DetReco)}

\subsection{Context du projet DetReco}

Cette mission s’inscrit dans le cadre d’une collaboration entre l’équipe STARS du centre de Sophia-Antipolis, la Direction Générale de l’Armement (DGA) et le département Défense et Sécurité de l’Inria.

L'objectif de ce projet consiste à mener une étude de l’état de l’art des algorithmes pouvant s’appliquer à la détection et à la reconnaissance, proche du temps réel, de véhicules militaires sur des images et vidéos; de proposer une implémentation de cet état de l’art fusionnant éventuellement plusieurs approches; ainsi que d’en faire une évaluation.
L’évaluation consistera non seulement en une quantification globale des performances mais aussi une mesure plus fine des erreurs de traitement (faux positifs, silence).


\subsection{Méthodologie}

Dans le cadre de ce projet, nous avons implémenté un algorithme qui s’appuie sur le modèle \textbf{YOLOv8} pour la détection et l’algorithme \textbf{DeepOCsort} pour le tracking de véhicules.
La méthodologie du projet DetReco repose sur plusieurs étapes clés :
\begin{enumerate}
	\item \textbf{Préparation des données :} Extraction et annotation d'images provenant de bases de données existantes. telles qu'ImageNet, OpenImages et Roboflow, complétées par des images collectées via des outils de scraping.
	      Ces images constituent le jeu de données de base pour l’entraînement de notre modèle de détection et de reconnaissance de véhicules militaires.

	\item \textbf{Entraînement du modèle} : Utilisation des jeux de données pour entraîner le modèle YOLOv8, qui est un modèle de détection d'objets en temps réel.
	      L'objectif est d'affiner le modèle pour qu'il soit capable de détecter et de reconnaître efficacement les véhicules militaires dans des séquences d’images et de vidéos.
	      L'entraînement permet d'améliorer la précision du modèle dans des environnements variés, avec des conditions de visibilité et des camouflages différents.

	\item \textbf{Suivi des objets} : Le modèle entraîné est ensuite appliqué pour détecter les véhicules militaires dans des vidéo.

	\item \textbf{Génération d’images} : Utilisation de techniques de génération d'images basées sur des modèles génératifs, pour créer des images supplémentaires de véhicules militaires dans diverses conditions (par exemple, sous différents angles, éclairages ou environnements).
	      Cette étape est indispensable pour enrichir le corpus d’entraînement car il manque des images pertinentes pour certains scénarios spécifiques.
	      En augmentant les images d'entraînement sous diverses conditions, les performances du modèle sur des situations réelles pourront s'accroître.

\end{enumerate}

\subsubsection{Définition des classes}

Afin de développer un modèle capable de faire la différence entre différents types de véhicules militaires, nous allons définir des classes larges en utilisant la catégorie \textit{Véhicules militaires par type} de Wikipédia.

Nous utiliserons 4 classes :

\begin{itemize}
	\item \textbf{Véhicule de combat blindé (AFV)}
	\item \textbf{Transport de troupes blindé (APC)}
	\item \textbf{Véhicule du génie militaire (MEV)}
	\item \textbf{Véhicule blindé léger (LAV)}
\end{itemize}

\subsubsection{Préparation des données}

\indent \textbf{Prepare-01 :} Nous commençons par créer un ensemble de données d'entraînement à partir d'images disponibles dans des ensembles de données de détection (dataset) d'objets open source. (ImageNet, OpenImages, Roboflow).

\begin{itemize}
	\item \textbf{ImageNet : }Le premier dataset que nous avons utilisé est ImageNet21k. De ce dataset, nous avons pu avoir 378 images annotées de tanks.
	\item \textbf{OpenImages : } Dans ce dataset, nous avons réussi à charger 1246 images annotées de tanks.
	\item \textbf{Roboflow : } Nous disposons désormais de 1624 images, mais cela reste insuffisant pour un entraînement optimal du modèle. Pour obtenir encore plus d'images d'entraînement, nous avons charger un autre dataset annotées de véhicules militaires.
\end{itemize}

\noindent Pour cette première étape, nous avons réussi à collecter \textbf{2666 images} de tanks dont \textbf{89 images} non annotées.\\


\noindent\textbf{Prepare-02 :} Nous utilisons également des outils de scraping pour collecter davantage d'images de véhicules militaires à partir d'images Google.

\noindent Nous avons collectés 669 images de plus, ce qui nous fait un total de \textbf{3335 images}.

Ce nombre d'images nous permet de définir quatre grandes classes de véhicules militaires que notre modèle peut ensuite discriminer : \textbf{Armoured Fighting Vehicle (AFV), Armoured Personnel Carrier (APC), Military Engineering Vehicle (MEV) and Light Armoured Vehicle (LAV)}.


\begin{figure}[H]
	\center
	\includegraphics[width=\textwidth]{./images/fiftyone-dataset.png}
	\caption[Fiftyone dataset]{FiftyOne - Dataset (3335 images)}\label{fig:fiftyone-dataset}
\end{figure}

\subsubsection{Entrainement du modèle}

Pour la détection de véhicules militaires, en utilisant les datasets d'images annotées exportés, nous allons utiliser le model de base \textbf{YoloV8}.

De base, nous utilisons le modèle \textbf{yolov8m.pt} (medium), qui offre un bon compromis entre taille et vitesse, mais d'autres modèles que nous utiliserons pour l'expérience, sont disponibles auprès \textit{d'ultralytics}.

\begin{table}[H]
	\centering
	\begin{tabular}{|p{2cm}|p{2cm}|p{1.7cm}|p{2.4cm}|p{2.4cm}|p{1.7cm}|p{1.7cm}|}
		\hline
		\textbf{Model}   & \textbf{Size (pixels)} & \textbf{mAP\textsuperscript{val} 50-95} & \textbf{Speed CPU ONNX (ms)} & \textbf{Speed A100 TensorRT (ms)} & \textbf{Params (M)} & \textbf{FLOPs (B)} \\ \hline
		\textit{YOLOv8n} & 640                    & 37.3                                    & 80.4                         & 0.99                              & 3.2                 & 8.7                \\ \hline
		\textit{YOLOv8s} & 640                    & 44.9                                    & 128.4                        & 1.20                              & 11.2                & 28.6               \\ \hline
		\textit{YOLOv8m} & 640                    & 50.2                                    & 234.7                        & 1.83                              & 25.9                & 78.9               \\ \hline
		\textit{YOLOv8l} & 640                    & 52.9                                    & 375.2                        & 2.39                              & 43.7                & 165.2              \\ \hline
		\textit{YOLOv8x} & 640                    & 53.9                                    & 479.1                        & 3.53                              & 68.2                & 257.8              \\ \hline
	\end{tabular}
	\caption{Comparaison des modèles YOLOv8}
	\label{tab:yolov8_comparaison}
\end{table}

A ce niveau de l'expérience, la collecte des données du dataset d'entraînement c'est faite en deux grande étapes (Prepare-01 et Prepare-02).
Nous allons présenter les résultats d'entraînement avec et sans images non annotées avec des modèles de base de différences tailles.

La précision (\textbf{mAP : Mean Average Precision}) de notre modèle sera relevé à la fin de chaque entraînement afin de suivre son évolution.

\begin{equation}
	\text{mAP} = \frac{1}{N} \sum_{i=1}^{N} \text{AP}_i
\end{equation}

\begin{itemize}
	\item \textbf{N} : Représente le nombre total de classes dans le dataset.
	\item \textbf{AP$_i$} : Représente l'Average Precision (Précision Moyenne) pour la $i$-ème classe.
	\item \textbf{mAP} : Représente la moyenne des précisions moyennes sur toutes les classes.
\end{itemize}



\subsubsection*{Prepare-01: ImageNet, OpenImages, Roboflow (2666 images)}

A cette étape, le modèle détecte uniquement une classe 'tank', donc la mAP sera une moyenne de toutes les images.

J’ai essayé plusieurs scénarios sur l’algorithme en changeant le paramètres \textit{Epoch} (époques: passe complète à travers toutes les instances de dataset d'entraînement.) et la taille du modèles.

\begin{table}[H]
	\centering
	\begin{tabular}{|c|c|c|c|}
		\hline
		\textbf{Modèles} & \textbf{Type d'images}                 & \textbf{Epoch} & \textbf{mAP} \\ \hline
		\multirow{7}{*}{\texttt{yolov8m}}
		                 & \multirow{3}{*}{Avec images négatives}
		                 & 60                                     & 0.661                         \\ \cline{3-4}
		                 &                                        & 80             & 0.655        \\ \cline{3-4}
		                 &                                        & 100            & 0.648        \\ \cline{2-4}
		                 & \multirow{4}{*}{Sans images négatives}
		                 & 60                                     & 0.714                         \\ \cline{3-4}
		                 &                                        & 70             & 0.692        \\ \cline{3-4}
		                 &                                        & 80             & 0.700        \\ \cline{3-4}
		                 &                                        & 100            & 0.700        \\ \hline

		\multirow{3}{*}{\texttt{yolov8l}}
		                 & \multirow{3}{*}{Avec images négative}
		                 & 60                                     & 0.651                         \\ \cline{3-4}
		                 &                                        & 80             & 0.649        \\ \cline{3-4}
		                 &                                        & 100            & 0.638        \\ \hline

		\multirow{3}{*}{\texttt{yolov8x}}
		                 & \multirow{3}{*}{Avec images négative}
		                 & 60                                     & 0.652                         \\ \cline{3-4}
		                 &                                        & 80             & 0.654        \\ \cline{3-4}
		                 &                                        & 100            & 0.631        \\ \hline
	\end{tabular}
	\caption{Résultats des entraînements à l'étape \textbf{Prepare-01(2666 images)}.}
	\label{tab:results_with_without_annotation}
\end{table}



Au vu de ces résultats, nous nous sommes rendu compte que la taille du modèle n'influence pas les résultats. Nous allons donc uniquement travailler avec la taille medium du modèle YoloV8.
Nous allons également utiliser des datasets contenant des images annotées.


\subsubsection*{Prepare-02: ImageNet, OpenImages, Roboflow and Google  (3335 images)}

A cette étape, nous avons entraîné le modèle a détecter les 04 classes (AFV, APC, MEV, LAV).
La moyenne mAP tient donc compte de ces 04 classes, mais nous allons aussi fournir la moyenne de chacune des classes.

\paragraph{\texttt{yolov8m :}}
- Epoch = 60 $\Rightarrow$ mAP = 0.6609
\begin{table}[h!]
	\centering
	\begin{tabular}{|l|c|c|c|c|}
		\hline
		Classe       & Précision & Rappel & F1-score & Support \\ \hline
		AFV          & 0.76      & 0.85   & 0.80     & 343     \\ \hline
		APC          & 0.69      & 0.72   & 0.70     & 64      \\ \hline
		LAV          & 0.55      & 0.72   & 0.62     & 25      \\ \hline
		MEV          & 0.62      & 0.71   & 0.67     & 7       \\ \hline
		micro avg    & 0.74      & 0.82   & 0.78     & 439     \\ \hline
		macro avg    & 0.65      & 0.75   & 0.70     & 439     \\ \hline
		weighted avg & 0.74      & 0.82   & 0.78     & 439     \\ \hline
	\end{tabular}
	\caption{Résultats pour \texttt{yolov8m} à Epoch 60}
	\label{tab:yolov8m_epoch60}
\end{table}




\subsubsection{Suivi des objets}

Après avoir entraîné notre model avec notre jeu de donnée, nous utilisons le meilleur model (\textbf{best.pt}) obtenu pendant l'entraînement et un tracker (\textbf{bytetrack.yaml}) pour tracker les objets dans les vidéos.


\subsubsection{Génération d'images :}

Nous avons utilisé des méthodes d'augmentation des données afin d'améliorer la capacité du modèle à reconnaître des véhicules même lorsqu'ils sont éloignés ou partiellement masqués. Les méthodes utilisées incluent :

\begin{itemize}
	\item Une transformation \textbf{scale} pour dézoomer l'image et donner l'impression que le véhicule est vu de loin. (Figure \ref{fig:scale_transformed})
	\item Une transformation \textbf{XYMasking} pour masquer des bandes horizontales ou verticales, simulant un véhicule caché par un obstacle. (Figure \ref{fig:xymasking})
	\item Une transformation de type \textbf{Météo} pour donner l'impression qu'il y a de la neige, de la pluie ou du brouillard sur l'image. (Figure \ref{fig:meteo})
\end{itemize}


\begin{figure}[H]
	\centering
	\begin{subfigure}[b]{0.40\textwidth}
		\centering
		\includegraphics[width=\textwidth]{./images/original-image.jpg}
		\caption{Image originale}
		\label{fig:original-image}
	\end{subfigure}
	\hfill
	\begin{subfigure}[b]{0.40\textwidth}
		\centering
		\includegraphics[width=\textwidth]{./images/scale_transformed.jpg}
		\caption{Effet zoom}
		\label{fig:scale_transformed}
	\end{subfigure}
	\vskip\baselineskip
	\begin{subfigure}[b]{0.40\textwidth}
		\centering
		\includegraphics[width=\textwidth]{./images/xymasking.jpg}
		\caption{XYMasking}
		\label{fig:xymasking}
	\end{subfigure}
	\hfill
	\begin{subfigure}[b]{0.40\textwidth}
		\centering
		\includegraphics[width=\textwidth]{./images/weather_effect.jpg}
		\caption{Météo (pluie)}
		\label{fig:meteo}
	\end{subfigure}
	\caption{Exemples transformation d'images de véhicules militaires}
	\label{fig:military-vehicles}
\end{figure}



\noindent Les étapes spécifiques suivies sont les suivantes :

\begin{enumerate}
	\item Dans un premier temps, nous avons récupéré une image de notre dataset avec ses labels (bounding boxes).
	\item Ensuite, nous avons créé une fonction qui applique les trois transformations à toutes les images du dataset.
	\item Enfin, nous avons entraîné le modèle YOLO avec ce dataset augmenté.
\end{enumerate}

Après l'augmentation des images, nous avons pu collecter en plus du dataset deja existant, \textbf{8403 images}.


\subsubsection*{Prepare-03: Dataset ImageNet, OpenImages, Russian Military Annotated (Roboflow), Google et données augmentées : 8403 images}

\begin{flushleft}
	Avant de commencer les analyses, il est important de noter que les performances du modèle dépendent fortement des paramètres d'entraînement, tels que le nombre d'époques et le type de données utilisées. Le tableau ci-dessous montre les résultats obtenus pour différents modèles et paramètres d'Epoch, en utilisant un ensemble de données combinant plusieurs sources.
\end{flushleft}

\begin{table}[H]
	\centering
	\begin{tabular}{|c|c|c|c|c|c|c|c|}
		\hline
		\multirow{2}{*}{\textbf{Modèles}} & \multirow{2}{*}{\textbf{Epoch}} & \multirow{2}{*}{\textbf{mAP}} & \multicolumn{4}{c|}{\textbf{Précision par classe}} & \multirow{2}{*}{\textbf{micro avg}}                      \\ \cline{4-7}

		                                  &                                 &                               & AFV                                                & APC                                 & LAV  & MEV  &      \\ \hline
		\multirow{5}{*}{\texttt{yolov8m}}
		                                  & 60                              & 0.343                         & 0.71                                               & 0.59                                & 0.51 & 0.22 & 0.68 \\ \cline{2-8}
		                                  & 70                              & 0.364                         & 0.71                                               & 0.57                                & 0.51 & 0.26 & 0.67 \\ \cline{2-8}
		                                  & 80                              & 0.405                         & 0.73                                               & 0.54                                & 0.51 & 0.21 & 0.67 \\ \cline{2-8}
		                                  & 90                              & 0.374                         & 0.72                                               & 0.63                                & 0.49 & 0.26 & 0.69 \\ \cline{2-8}
		                                  & 100                             & 0.327                         & 0.67                                               & 0.49                                & 0.60 & 0.27 & 0.64 \\ \hline
		\multirow{2}{*}{\texttt{yolov8l}}
		                                  & 80                              & 0.360                         & 0.69                                               & 0.53                                & 0.45 & 0.24 & 0.65 \\ \cline{2-8}
		                                  & 100                             & 0.331                         & 0.69                                               & 0.63                                & 0.52 & 0.21 & 0.67 \\ \hline
	\end{tabular}
	\caption{Résultats des entrainements à l'étapes \textbf{Prepare-03 (8403 images)}.}
	\label{tab:results}
\end{table}

\begin{flushleft}
	Les résultats révèlent que le modèle \texttt{yolov8m} obtient ses performances optimales après 80 époques, moment où le mAP atteint son niveau le plus élevé. Au-delà de ce seuil, une augmentation supplémentaire du nombre d'époques n'améliore pas nécessairement les résultats, ce qui met en évidence la nécessité de trouver un juste équilibre entre sous-apprentissage et surapprentissage.
\end{flushleft}


