\markboth{\MakeUppercase{Conclusion}}{}
\addcontentsline{toc}{chapter}{Conclusion}
\sloppy

\textbf{Conclusion Provisoire}\\


Ce mémoire a permis d'explorer l'utilisation des algorithmes de deep learning pour la détection et la reconnaissance en temps réel de véhicules militaires dans des images et des vidéos.
À travers l'étude de l'état de l'art et la contribution au projet ADOMVI, plusieurs avancées ont été réalisées, notamment l'optimisation des modèles de détection comme YoloV8 et l'intégration de techniques de data augmentation, telles que l'utilisation de modèles génératifs comme Stable Diffusion.

L'analyse a montré que, bien que les méthodes actuelles offrent des résultats prometteurs, elles présentent encore certaines limites, notamment en termes de robustesse face à des conditions de visibilité dégradées ou de rareté des données d'entraînement.
Ces résultats soulignent l'importance de continuer à explorer des méthodes plus sophistiquées et à améliorer les jeux de données pour mieux répondre aux contraintes spécifiques du domaine militaire.

En conclusion, ce travail a posé les bases pour des recherches futures dans le domaine de la détection et de la reconnaissance en temps réel, en mettant en lumière l'importance des données de qualité et des techniques avancées de deep learning pour répondre aux exigences opérationnelles des acteurs de la défense.
Il ouvre également la voie à des collaborations renforcées entre les centres de recherche et les entités de défense pour le développement de solutions technologiques innovantes et adaptées.

