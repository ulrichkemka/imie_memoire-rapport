\markboth{\MakeUppercase{Abstract}}{}
\addcontentsline{toc}{chapter}{Abstract}
\sloppy

Real-time detection and recognition of military vehicles represent major challenges in the defense domain. These tasks aim to improve the accuracy of detection systems while addressing the specific constraints of military operations, such as the scarcity of training data and difficult visibility conditions.

In this study, we focus on optimizing deep learning models for object detection, particularly the detection of military vehicles from images and videos. The DetReco project, to which this work contributed, aims to enhance the performance of detection models like YoloV8 by integrating techniques such as generative models and data augmentation.

To this end, we proposed improvements based on data augmentation techniques and the use of generative models to simulate various scenarios and enrich the dataset. Our experiments show significant results in terms of detection performance improvement, particularly in complex situations. However, our results also highlight the limitations related to the robustness of algorithms in real-world conditions.

We conclude that further efforts must be invested in customizing models to meet the specific requirements of military operations, while exploring new approaches to overcome the challenges posed by data scarcity and visibility conditions.
