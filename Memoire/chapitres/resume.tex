\markboth{\MakeUppercase{Résumé}}{}
\addcontentsline{toc}{chapter}{Résumé}
\sloppy

La détection et la reconnaissance en temps réel de véhicules militaires constituent des défis majeurs dans le domaine de la défense. Ces tâches visent à améliorer la précision des systèmes de détection, tout en répondant aux contraintes spécifiques des opérations militaires, telles que la rareté des données d'entraînement et les conditions de visibilité difficiles.

Dans cette étude, nous nous intéressons à l’optimisation des modèles de deep learning pour la détection d’objets, en particulier la détection de véhicules militaires à partir d’images et de vidéos. Le projet DetReco, auquel ce travail a contribué, a pour objectif d’améliorer les performances des modèles de détection comme YoloV8, grâce à l'intégration de techniques telles que les modèles génératifs et l'augmentation de données.

Dans cette optique, nous avons proposé des améliorations basées sur les techniques d’augmentation de données et l’utilisation de modèles génératifs pour simuler des scénarios variés et enrichir le jeu de données. Nos expériences montrent des résultats significatifs en termes d'amélioration des performances de détection, notamment dans des situations complexes. Cependant, nos résultats soulignent également les limites liées à la robustesse des algorithmes dans des conditions réelles.

Nous concluons que des efforts supplémentaires doivent être investis dans la personnalisation des modèles pour les adapter aux exigences spécifiques des opérations militaires, tout en explorant de nouvelles approches pour pallier les lacunes liées à la rareté des données et aux conditions de visibilité.