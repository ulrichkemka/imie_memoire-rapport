\markboth{\MakeUppercase{Résumé}}{}
\addcontentsline{toc}{chapter}{Résumé}
\sloppy

\textbf{Résumé Provisoire}\\

Ce mémoire traite de l'optimisation des algorithmes de deep learning pour la détection et la reconnaissance en temps réel de véhicules militaires dans des images et des vidéos.
L'objectif principal est d'étudier l'état de l'art des méthodes existantes et d'apporter des améliorations basées sur l'intégration de techniques avancées telles que les modèles génératifs et la data augmentation.
Le projet ADOMVI, auquel ce travail a contribué, vise à optimiser les performances des modèles de détection comme YoloV8, en tenant compte des contraintes spécifiques du domaine militaire, notamment la rareté des données et les conditions de visibilité variées.
Les résultats obtenus montrent une amélioration notable des performances de détection et de reconnaissance, tout en soulignant les défis persistants liés à la robustesse des modèles en conditions réelles.
Ce mémoire conclut sur la nécessité de poursuivre les recherches pour perfectionner ces algorithmes et adapter les solutions technologiques aux besoins des opérations de défense.

