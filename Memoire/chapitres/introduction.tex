\markboth{\MakeUppercase{Introduction}}{}
\addcontentsline{toc}{chapter}{Introduction}
\sloppy

\textbf{Introduction Provisoire}\\

Aujourd'hui, les données sont souvent comparées au pétrole du XXIe siècle. 
Chaque jour, plusieurs téraoctets de données sont stockés, formant ainsi le carburant essentiel de la technologie moderne et de l'intelligence artificielle (IA) en particulier.
Avec une telle abondance, il semble facile de trouver suffisamment de données pour le développement de l'IA.
Cependant, de nombreux organismes, issus de secteurs variés, rencontrent des difficultés pour accéder à certaines catégories d’informations indispensables à l'évolution de leur IA.
Ces difficultés sont principalement dues à la rareté et à la confidentialité des données dans certains secteurs, notamment dans le domaine de la vision optique.

Dans ce contexte, nous nous sommes interrogés sur la manière dont le deep learning peut améliorer la détection et la reconnaissance en temps réel de véhicules militaires dans des images et vidéos.
L'objectif de ce mémoire est d'étudier l'état de l'art des méthodes et algorithmes appliqués à cette tâche, afin de formuler des recommandations.

Ce travail s'inscrit dans le cadre du projet DetReco (Détection et reconnaissance de véhicules militaires sur des images et vidéos), un projet visant à optimiser les algorithmes de deep learning pour la détection et la reconnaissance en temps réel de véhicules militaires.
Ce projet repose sur la création et l'amélioration de jeux de données spécifiques, ainsi que sur le fine-tuning de modèles de détection d'objets. L'intégration de modèles génératifs, pour augmenter la diversité des données d'entraînement, a également constitué une part importante de ce projet.